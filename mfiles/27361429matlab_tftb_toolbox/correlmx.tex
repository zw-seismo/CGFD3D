\markright{correlmx}
\hspace*{-1.6cm}{\Large \bf correlmx}

\vspace*{-.4cm}
\hspace*{-1.6cm}\rule[0in]{16.5cm}{.02cm}
\vspace*{.2cm}

{\bf \large \fontfamily{cmss}\selectfont Purpose}\\
\hspace*{1.5cm}
\begin{minipage}[t]{13.5cm}
computes an estimation of the autocorrelation matrix of a signal.
\end{minipage}
\vspace*{.2cm}

{\bf \large \fontfamily{cmss}\selectfont Synopsis}\\
\hspace*{1.5cm}
\begin{minipage}[t]{13.5cm}
\begin{verbatim}
Rx=correlmx(x,p,Rxtype)
Rx=correlmx(x,p)
\end{verbatim}
\end{minipage}
\vspace*{.2cm}

{\bf \large \fontfamily{cmss}\selectfont Description}\\
\hspace*{1.5cm}
\begin{minipage}[t]{13.5cm}
{\ty correlmx} computes a correlation matrix of a signal. 

\begin{tabular*}{14cm}{p{1.5cm} p{8.5cm} c}
Name & Description & Default value\\\hline
{\ty x}      & analyzed signal &  \\ 
{\ty p}      & highest autocorrelation lag. For a given $p$, 
               the size of the autocorrelation matrix will be 
               $(p+1)\times(p+1)$ & \\
{\ty Rxtype} & computation algorithm. 
               Possible values for 
               \texttt{Rxtype} are \texttt{'burg'}, 
               \texttt{'hermitian'} and \texttt{'fbhermitian'}
             & {\ty 'fbhermitian'}\\\hline 
{\ty Rx}     & computed correlation matrix, of size \hbox{$(p+1)\times(p+1)$} & \\\hline 
\end{tabular*}

For a given signal $x[n]$, $n\,\in\,[0,N-1]$, this function
computes \texttt{Rx} by different approaches~:

If \texttt{Rxtype} equals \texttt{'burg'}, then \texttt{correlmx}
first computes the autocorrelation function
$$
R_x[\tau]=
\left\{{
\begin{array}{lcl}
{1\over N-\tau}\,\sum_{n=\tau}^{N-1}x[n]\,x[n-\tau]^*
&\quad{\rm if}\quad& \tau\geq 0\\[0.2 cm]
{1\over N+\tau}\,\sum_{n=0}^{N+\tau-1}x[n]\,x[n-\tau]^*
&\quad{\rm if}\quad& \tau< 0
\end{array}
}\right.
$$
for $\tau\,\in\,[-p,p]$, and then builds a toeplitz matrix
\texttt{Rx} such that $(R_x)_{ij}=R_x[i-j]$.

If \texttt{Rxtype} equals \texttt{'hermitian'} or 
\texttt{'fbhermitian'}, then \texttt{correlmx} computes
$$
R_x^h={1\over N-p}\,\sum_{n=p}^{N-1}X[n]^*\,X[n]^T
$$
with $X[n]^T=\left[{x[n]\ x[n-1]\ \ldots\ x[n-p]}\right]$.
If \texttt{Rxtype} equals \texttt{'hermitian'}, then 
\texttt{Rxtype} returns $R_x^h$, and if \texttt{Rxtype} equals 
\texttt{'fbhermitian'}, then \texttt{Rxtype} returns 
$0.5\,(R_x^h+{R_x^h}^H)$.
\end{minipage}

\newpage
\vspace*{1cm}
{\bf \large \fontfamily{cmss}\selectfont Example}\\
\hspace*{1.5cm}
\begin{minipage}[t]{13.5cm}

\begin{verbatim}
N=100; sig=real(fmconst(N,0.1))+0.4*randn(N,1); 
Rx=correlmx(sig,2,'burg'); [v,d] = eig(Rx)
acos(-0.5*v(2,1)/v(1,1))/(2*pi)
Rx=correlmx(sig,2,'hermitian'); [v,d] = eig(Rx)
acos(-0.5*v(2,1)/v(1,1))/(2*pi)
\end{verbatim}
computes an estimation of the frequency of a sinusoid using the Pisarenko
approach, with a correlation matrix computed by the Burg or the hermitian
algorithm.

\end{minipage}
\vspace*{.5cm}


{\bf \large \fontfamily{cmss}\selectfont See Also}\\
\hspace*{1.5cm}
\begin{minipage}[t]{13.5cm}
\begin{verbatim}
parafrep, tfrparam
\end{verbatim}
\end{minipage}
