\chapter{Getting started}\label{chapter-run}
This chapter only gives a brief introduction to the running of the program and the configuration of the main parameter file and running script. For more detailed information about these files, please refer to concerning chapter.
\section{Running CGFD3D}
The program can be running by simply using the \texttt{sh} command, eg. for acoustic wave modeling
\begin{lstlisting}[language = bash]
sh cgfd3d-wave-ac.sh
\end{lstlisting}
The parameters and output directions can be modified in the \texttt{.sh} file.
\section{Parameters in Main Par File}
\begin{itemize} 
\item Number of grid points.
\begin{lstlisting}[language=json,
 frame=tb]
 "number_of_total_grid_points_x" : 300,
 "number_of_total_grid_points_y" : 300,
 "number_of_total_grid_points_z" : 60,                
\end{lstlisting}
\item Number of mpi processes.
\begin{lstlisting}[language=json,
 frame=tb]
 "number_of_mpiprocs_x" : 3,
 "number_of_mpiprocs_y" : 3,                
\end{lstlisting}
\item Size and number of time steps or simply the length of time window, which will automatically give a stable time step.
\begin{lstlisting}[language=json,
 frame=tb]
  "size_of_time_step" : 0.008,
  "#size_of_time_step" : 0.020,
  "number_of_time_steps" : 500,
  "#time_window_length" : 4,
  "check_stability" : 1,                
\end{lstlisting}
\item Absorbing boundary parameters.
\begin{lstlisting}[language=json,
 frame=tb]
 "boundary_x_left" : {
       "cfspml" : {
           "number_of_layers" : 10,
           "alpha_max" : 3.14,
           "beta_max" : 2.0,
           "ref_vel"  : 7000.0
           }
       },
  ....
  "boundary_z_top" : {
      "free" : "timg"
        },                
\end{lstlisting}
\item The grid configuration, which can be imported from outside or use the default cartesian coordinate system.
\begin{lstlisting}[language=json,
 frame=tb]    
"grid_generation_method" : {
      "#import" : "$GRID_DIR",
      "cartesian" : {
      "origin"  : [0.0, 0.0, -5900.0 ],
      "inteval" : [ 100.0, 100.0, 100.0 ]
      },
      "#layer_interp" : {
       "in_grid_layer_file" : "$INPUTDIR/prep_grid/tangshan_area_topo.gdlay",
       "refine_factor" : [ 1, 1, 1 ],
       "horizontal_start_index" : [ 3, 3 ],
       "vertical_last_to_top" : 0
      }
  },
  "is_export_grid" : 1,
  "grid_export_dir"   : "$GRID_DIR",
\end{lstlisting}
\item The method used to calculate the metric parameters.
\begin{lstlisting}[language=json,
 frame=tb]    
 "metric_calculation_method" : {
       "#import" : "$GRID_DIR",
       "calculate" : 1
   },
 "is_export_metric" : 1,
\end{lstlisting}
\item The medium configuration.
\begin{lstlisting}[language=json,
 frame=tb]    
 "medium" : {
       "type" : "elastic_iso",
       "input_way" : "infile_layer",
       "#input_way" : "binfile",
       "#binfile" : {
         "size"    : [1101, 1447, 1252],
         "spacing" : [-10, 10, 10],
         "origin"  : [0.0,0.0,0.0],
         "dim1" : "z",
         "dim2" : "x",
         "dim3" : "y",
         "Vp" : "$INPUTDIR/prep_medium/seam_Vp.bin",
         "Vs" : "$INPUTDIR/prep_medium/seam_Vs.bin",
         "rho" : "$INPUTDIR/prep_medium/seam_rho.bin"
       },
       "code" : "func_name_here",
       "import" : "$MEDIA_DIR",
       "infile_layer" : "$INPUTDIR/prep_medium/basin_el_iso.md3lay",
       "infile_grid" : "$INPUTDIR/prep_medium/topolay_el_iso.md3grd",
       "equivalent_medium_method" : "loc",
       "#equivalent_medium_method" : "har"
   },
   "is_export_media" : 1,
   "media_export_dir"  : "$MEDIA_DIR",
\end{lstlisting}
\item The viscosity configuration.

\begin{lstlisting}[language=json,
 frame=tb]    
 "#visco_config" : {
       "type" : "graves_Qs",
       "Qs_freq" : 1.0
   },
\end{lstlisting}
\item The input source file.
\begin{lstlisting}[language=json,
 frame=tb]    
 "in_source_file" : "$INPUTDIR/event_3moment_discrete.src",
 "is_export_source" : 1,
 "source_export_dir"  : "$SOURCE_DIR",
\end{lstlisting}
\item The station list file.
\begin{lstlisting}[language=json,
 frame=tb]    
"in_station_file" : "$INPUTDIR/station.list",
 
\end{lstlisting}
\item The output data that needs to be stored, including the receiver line, the slice and the snapshot.
\begin{lstlisting}[language=json,
 frame=tb]    
"receiver_line" : [
    {
      "name" : "line_x_1",
      "grid_index_start"    : [  0, 49, 59 ],
      "grid_index_incre"    : [  1,  0,  0 ],
      "grid_index_count"    : 20
    },
    {
      "name" : "line_y_1",
      "grid_index_start"    : [ 19, 49, 59 ],
      "grid_index_incre"    : [  0,  1,  0 ],
      "grid_index_count"    : 20
    } 
  ],

  "slice" : {
      "x_index" : [ 190 ],
      "y_index" : [ 90 ],
      "z_index" : [ 59 ]
  },

  "snapshot" : [
    {
      "name" : "volume_vel",
      "grid_index_start" : [ 0, 0, 59 ],
      "grid_index_count" : [ 300,300, 1 ],
      "grid_index_incre" : [  1, 1, 1 ],
      "time_index_start" : 0,
      "time_index_incre" : 1,
      "save_velocity" : 1,
      "save_stress"   : 1,
      "save_strain"   : 1
    }
  ], 
\end{lstlisting}
\end{itemize}
\section{Parameters in Running Script}
The running script mainly contains the initialization of the environment and the make command. The initialization makes sure that the desired versions of programs are added to the environment viriables. Environment Mudules package is a tool that simplify shell initialization and allows users to manage different versions of applictions with faster compiling speed. The export command can be used to temporarily initiate the environment if the Environment Modules packages are not installed. Both commands are optional and you need to modify the path corresponding to your current path of installation.    
\begin{lstlisting}[language=bash,
    label={run_make_sh},
    caption=The running script in .json,
    frame=tb]
#!/bin/bash

#-- use module to set mpicc etc, on Mars
#module load intel/2019.5
#module load mpi/mpich/3.4.1_intel_2019.5
#module load mpi/mpich/3.3.1_intel_2019.5
#module load netcdf-c/4.4.1

#-- add mpi to PATH if module is not used, on server1
MPI_ROOT=/share/apps/gnu-4.8.5/mpich-3.3/
export PATH=$MPI_ROOT/bin:$PATH
exprot NETCDFROOT=/share/apps/gnu-4.8.5/disable-netcdf-4.4.1

echo "mpicc and mpicxx will invoke:"
mpicc -show
mpicxx -show

echo
echo "start to make ..."
make -f Makefile 2>&1 | tee log.make
\end{lstlisting}