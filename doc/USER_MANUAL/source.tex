\chapter{Source}\label{chapter-source}

\begin{lstlisting}[language=python, title=run\_test.sh, frame=tb]
    "source_input" : {
        "in_par" : {
            "name" : "evt_by_par"
            "source" : [
                {
                    "#index" : [40, 50, 30],
                    "coord" : [4000, 5000, -2900],
                    "wavelet_name" : "ricker",
                    "ricker_center_frequency" : 0.5,
                    "ricker_peak_time" : 2.0,
                    "#wavelet_name" : "gaussian",
                    "#gaussian_rms_width" : 2.0,
                    "#gaussian_peak_time" : 0.5,
                    "start_time" : 0.0,
                    "end_time" : 4.0,
                    "force_vector" : [1.0e13, 1.0e10, 1.0e11],
                    "moment_tensor" : [1.2e13, 1.0e10, 1.0e14, 1.0e12, 1.0e11, 1.0e12]
                },
                {
                    "#index" : [30, 20, 30],
                    "coord" : [3000, 3000, -2900],
                    "wavelet_name" : "ricker",
                    "ricker_center_frequency" : 0.5,
                    "ricker_peak_time" : 2.0,
                    "#wavelet_name" : "gaussian",
                    "#gaussian_rms_width" : 2.0,
                    "#gaussian_peak_time" : 0.5,
                    "start_time" : 0.0,
                    "end_time" : 4.0,
                    "force_vector" : [1.0e13, 1.0e10, 1.0e11],
                    "moment_tensor" : [1.2e13, 1.0e10, 1.0e14, 1.0e12, 1.0e11, 1.0e12]
                }
            ]
        },
        "#in_source_file" : "$INPUTDIR/source.anasrc"   #could choose source.valsrc
  },
  "is_export_source" : 1,                         # if export source                   
  "source_export_dir"  : "$SOURCE_DIR",                   
\end{lstlisting}

%=============================================================
\section{JSON read source format} \label{JSON read source format}

In the configuration of source, if \texttt{in\_par} is selected, that means the source information is read by JSON. \texttt{index} is source grid index location and \texttt{coord} is source distance location, Input order is x, y, z. Select one of the two representations to determine the location of the source. \texttt{wavelet\_name} include ricker, ricker\_deriv, gaussian, gaussain\_deriv. if wavelet\_name choose ricker or ricker\_deriv, should give \texttt{ricker\_center\_frequency} and \texttt{ricker\_peak\_time}. if wavelet\_name choose gaussian or gaussian\_deriv, should give \texttt{gaussian\_rms\_width} and \texttt{gaussian\_peak\_time}. \texttt{start\_time} is source function start time, and \texttt{end\_time} is source function end time. \texttt{force\_vector} is single force source component, the input order is Fx, Fy, Fz. \texttt{moment\_tentor} is double couple source component, the input order is Mxx, Myy, Mzz, Myz, Mxz, Mxy. \texttt{force\_vector} and \texttt{moment\_tentor} could choose together.
if \texttt{in\_source\_file} is selected, there need to be a .anasrc file or .valsrc file, and the .anasrc file format is shown in Section \ref{analytic source format}. The .valsrc file format is shown in Section \ref{value source format}.

%===================================================================
\section{analytic source format} \label{analytic source format}

A .anasrc file can be given as:

\begin{lstlisting}[language=python, title=source.anasrc, frame=tb]
test_event_1
3 2
4.0 
3000.0 2000.0 -3000.0
1.0e15 1.0e13 1.0e14
ricker
2.0 0.5
0.0
1000.0 2000.0 -3000.0
1.0e15 1.0e13 1.0e14
gaussian
2.0 0.5
1.0
1000.0 2000.0 -3000.0
1.0e15 1.0e13 1.0e14
gaussian_deriv
1.0 5.0
1.0
5000.0 2000.0 -2000.0
moment_tensor
1.0e15 1.0e13 1.0e14 1.0e15 1.0e13 1.0e14
gaussian_deriv
1.0 5.0
0.0
1500.0 2200.0 -1300.0
mechanism_angle
80.0 70.0 30.0 1.1e10 1.1 1.0e8 
ricker_deriv
2.0 0.5
2.0
\end{lstlisting}

first line \texttt{test\_event\_1} is source name. second line \texttt{3 2} is force source number 3 and moment source number 2. third line \texttt{4.0} is source function time length, be equal to end\_time subtract start\_time. All source functions have the same time length. Next lines, the information of each source is given in order. 

A force source, first line give coordinates \texttt{3000.0 2000.0 -3000.0}, input order is x, y, z. Second line give force\_vector \texttt{1.0e15 1.0e13 1.0e14}, input order is Fx, Fy, Fz. Third line is wavelet name, include ricker, ricker\_deriv, gaussian, gaussian\_deriv. The first force source wavelet name is ricker. Fourth line is ricker\_center\_frequency and ricker\_peak\_time, respectively, if wavelet name is ricker or ricker\_deriv. Fourth line is gaussian\_rms\_width and gaussian\_peak\_time, respectively, if wavelet name is gaussian or gaussian\_deriv. The first force source ricker\_center\_frequency is 2.0, ricker\_peak\_time is 0.5. The fifth line is source function start time, The first force source start time is 0.

A moment source, first line give coordinates. The first moment source coordinate is \texttt{5000.0 2000.0 -2000.0}, input order is x, y, z. Second line give moment source representation type, include moment\_tensor and mechanism\_angle. The first moment source representation is \texttt{moment\_tensor}, so third line give monmen\_tensor component, input order is Mxx, Myy, Mzz, Myz, Mxz, Mxy. Mxx = 1.01e15, Myy = 1.0e13, Mzz = 1.0e14, Myz = 1.0e15, Mxz = 1.0e13, Mxy = 1.0e14. The fourth line is wavelet name, include ricker\_deriv, gaussian\_deriv. The first moment source wavelet name is gaussian\_deriv. The fith line is gaussian\_rms\_width 1.0 and gaussian\_peak\_time 5.0. The sixth line is start time 0.0.

The second moment source representation type is \texttt{mechanism\_angle}, so next line input strike dip rake u D A. u is shear modulus, D is slip distance, unit is m, A is area. Strike = 80.0,  dip = 70.0, rake = 30.0, u = 1.1e10, D = 1.1, A = 1.0e8.   

%=============================================================
\section{value source format} \label{value source format}

A .valsrc file can be given as:

\begin{lstlisting}[language=python, title=source.valsrc, frame=tb]
test_event_1
1 1
0.02 20 
3000.0 2000.0 -3000.0
2000.0 3000.0 -2000.0
1.0
1407.403198 14074.032227 140740.312500
5305.979492 53059.796875 530597.937500
19359.816406 193598.187500 1935981.750000
68359.867188 683598.687500 6835986.500000
233580.953125 2335809.750000 23358096.000000
772291.812500 7722918.500000 77229184.000000
2470593.750000 24705938.000000 247059376.000000
7646484.000000 76464848.000000 764648448.000000
22894096.000000 228940960.000000 2289409536.000000
66305560.000000 663055616.000000 6630556160.000000
185734304.000000 1857343104.000000 18573430784.000000
503157920.000000 5031579648.000000 50315792384.000000
1318030848.000000 13180308480.000000 131803086848.000000
3338105344.000000 33381054464.000000 333810532352.000000
8172575744.000000 81725759488.000000 817257578496.000000
19338747904.000000 193387479040.000000 1933874692096.000000
44220518400.000000 442205208576.000000 4422052085760.000000
97689878528.000000 976898818048.000000 9768988049408.000000
208449208320.000000 2084492148736.000000 20844921225216.000000
429488177152.000000 4294882099200.000000 42948817321984.000000
1.5
moment_tensor
15145.9 15145.9 15145.9 15145.9 15145.9 15145.9
28607.7 28607.7 28607.7 28607.7 28607.7 28607.7
51960.8 51960.8 51960.8 51960.8 51960.8 51960.8
90666.5 90666.5 90666.5 90666.5 90666.5 90666.5
151799.5 151799.5 151799.5 151799.5 151799.5 151799.5
243494.3 243494.3 243494.3 243494.3 243494.3 243494.3
373476.1 373476.1 373476.1 373476.1 373476.1 373476.1
546371.6 546371.6 546371.6 546371.6 546371.6 546371.6
759746.0 759746.0 759746.0 759746.0 759746.0 759746.0
999278.3 999278.3 999278.3 999278.3 999278.3 999278.3
1234194.8 1234194.8 1234194.8 1234194.8 1234194.8 1234194.8
1414798.1 1414798.1 1414798.1 1414798.1 1414798.1 1414798.1
1474371.2 1474371.2 1474371.2 1474371.2 1474371.2 1474371.2
1337506.4 1337506.4 1337506.4 1337506.4 1337506.4 1337506.4
935643.3 935643.3 935643.3 935643.3 935643.3 935643.3
228314.9 228314.9 228314.9 228314.9 228314.9 228314.9
-774273.6 -774273.6 -774273.6 -774273.6 -774273.6 -774273.6
-1994150.0 -1994150.0 -1994150.0 -1994150.0 -1994150.0 -1994150.0
-3281599.0 -3281599.0 -3281599.0 -3281599.0 -3281599.0 -3281599.0
-4430085.0 -4430085.0 -4430085.0 -4430085.0 -4430085.0 -4430085.0
\end{lstlisting}

first line \texttt{test\_event\_1} is source name. second line \texttt{1 1} is force source number 1 and moment source number 1. Third line \texttt{0.02 20} is input value source time step interval 0.02 and time steps 20. Fourth line is force source coordinate \texttt{3000.0 2000.0 -3000.0}, Fifth line is moment source coordinate \texttt{2000.0 3000.0 -2000.0}. Next lines, the values of each source is given in order.

A force source, first give start time. The first force start time is \texttt{1.0}. Then give each time step force vector, each line input order is Fx, Fy, Fz.

A moment source, first give start time. The first moment start time is \texttt{1.5}. The second line give moment source representation type, include moment\_tensor and mechanism\_angle. If moment source representation is \texttt{moment\_tensor}, next each line give moment\_tensor component, input order is Mxx, Myy, Mzz, Myz, Mxz, Mxy. If moment source representation is \texttt{mechanism\_angle}, next each line give mechanism\_angle component, input order is strike, dip, rake, u, D, A. 

