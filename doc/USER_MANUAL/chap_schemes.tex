\chapter{Appendix B}\label{appd_schemes}
There are mainly staggered gird and collocated grid to solve the system of 1st-oeder velocity-stress equations in finite difference numerical simulation.In the staggered,velocity component and stress component separate by half a mesh, while in collocated grid velocity and stress locate at the same mesh. Because of the high calculation efficiency and small truncation error, in this work, staggered gird finite difference is used to solve cartesian grid cases. However, in curve grid, the spatial derivatives along three directions are required for components. So interpolation is needed. And the calculation accuracy and efficiency will decrese due to the interpolation. Thus,collocated grid finite difference method is used for curve grid cases in this work. The following introduces the scheme used in the code.
%=============================================================
\section{Collocated-grid DRP/opt MacCormack Scheme} 
%=============================================================
MacCormack scheme realizes the dissipation of non-physical high frequency components of the wave field by splitting the central difference into forward one-sided difference and backward ont-sided difference. It does not require explicit filter and can achieve truncation error of central difference.\cite{hixon1997increasing} combined MacCormack schemes with DRP(Dispersion Relation Preserving)(\cite{tam1993dispersion}),developed DRP/opt MacCormack schemes of high precision. The schemes greatly improved mesh resolution.
%=============================================================
\subsection{Space Discretization} 
%=============================================================
For 3D 1st-order partial differential equations like
\begin{align}
	W=A\frac{\partial W}{\partial \xi}+B\frac{\partial W}{\partial \eta}+C\frac{\partial W}{\partial \zeta}
\end{align}
The one-sided difference operators:
\begin{align}
	\hat{W}_{i}^{F} &= \frac{1}{\triangle \eta} \sum_{k=-1}^{k=3} a_{n} W_{i+k} \\ 
	\hat{W}_{i}^{B} &= \frac{1}{\triangle \eta} \sum_{k=-1}^{k=3}-a_{n} W_{i-k}
\end{align}
where $\hat{W}_i^F$and$\hat{W}_i^B$denote the forward and backward difference operatos with respect to $x,y,z$, coefficients are
\begin{subequations}
\begin{align}
	a_{-1} &= -0.30874, \\
	a_{0} &= -0.6326, \\
	a_{1} &= 1.2330, \\
	a_{2} &= -0.3334, \\
	a_{3} &= 0.04168.
\end{align} 
\end{subequations}
The stencil width is 7 grid points,when close to the boundary, the difference scheme with a smaller stencil width is used.\\ \\
3 gird points MacCormack difference scheme:
	\begin{align}
		\hat{W}_{i}^{F} &= \frac{1}{\triangle \eta} \sum_{k=-1}^{k=1} a_{n} W_{i+k} \\ 
		\hat{W}_{i}^{B} &= \frac{1}{\triangle \eta} \sum_{k=-1}^{k=1}-a_{n} W_{i-k}  
	\end{align}
where coefficients are
\begin{subequations}
\begin{align}
	a_{0} &= -1.0, \\
	a_{1} &= 1.0. \\
\end{align}  
\end{subequations}
5 grid points MacCormack difference scheme:
\begin{align}
	\hat{W}_{i}^{F} &= \frac{1}{\triangle \eta} \sum_{k=-1}^{k=2} a_{n} W_{i+k} \\ 
	\hat{W}_{i}^{B} &= \frac{1}{\triangle \eta} \sum_{k=-1}^{k=2}-a_{n} W_{i-k}  
\end{align}
where coefficients are 
\begin{subequations}
\begin{align}
	a_{-1} &= - \frac{7}{6}, \\
	a_{0} &= \frac{8}{6}, \\
	a_{1} &= -\frac{1}{6}.
\end{align} 
\end{subequations}
Here, 8 one-sided difference combinations are used.
\begin{subequations}
\begin{align}
	\hat{L}^{BBB} &= A \hat{W}^{B} + B\hat{W}^{B} + C\hat{W}^{B}, \\
	\hat{L}^{FFB} &= A \hat{W}^{F} + B\hat{W}^{F} + C\hat{W}^{B}, \\
	\hat{L}^{FFF} &= A \hat{W}^{F} + B\hat{W}^{F} + C\hat{W}^{F}, \\
	\hat{L}^{BBF} &= A \hat{W}^{B} + B\hat{W}^{B} + C\hat{W}^{F}, \\
	\hat{L}^{BFB} &= A \hat{W}^{B} + B\hat{W}^{F} + C\hat{W}^{B}, \\
	\hat{L}^{FBB} &= A \hat{W}^{F} + B\hat{W}^{B} + C\hat{W}^{B}, \\
	\hat{L}^{FBF} &= A \hat{W}^{F} + B\hat{W}^{B} + C\hat{W}^{F}, \\
	\hat{L}^{BFF} &= A \hat{W}^{B} + B\hat{W}^{F} + C\hat{W}^{F}, 
\end{align}
\end{subequations}
%=============================================================
\subsection{Time Discretization} 
%=============================================================
Multi-step and high-order Runge-Kutta is popular in partial differential numerical solutions(\cite{hu1996low,Zhang2012three}).Here, four stages Runge-Kutta scheme is used.The following is the steps.
\begin{subequations}
\begin{align}
	h^{(1)} &= \vartriangle t \hat{L}^{FFF}(W^{n}), \\
	h^{(2)} &= \vartriangle t \hat{L}^{BBB}(W^{n} + \alpha_{2} h^{(1)}), \\
	h^{(3)} &= \vartriangle t \hat{L}^{FFF}(W^{n} + \alpha_{3} h^{(2)}), \\
	h^{(4)} &= \vartriangle t \hat{L}^{BBB}(W^{n} + \alpha_{4} h^{(3)}), \\
	W^{n+1} &= W^{n} + \beta_{1} h^{(1)}+ \beta_{2} h^{(2)}+ \beta_{3} h^{(3)}+ \beta_{4} h^{(4)}, 
\end{align}\label{eqn:RK}
\end{subequations}
equation\ref{eqn:RK} is abbreviated as $W^{n+1} = \hat{L}^{FFF} \hat{L}^{BBB} \hat{L}^{FFF} \hat{L}^{BBB} W^{n}$, where $\hat{L}^{FFF} = A\hat{W}^{F}+B\hat{W}^{F}+C\hat{W}^{F}$, $\hat{L}^{BBB} = A\hat{W}^{B}+B\hat{W}^{B}+C\hat{W}^{B}$, $\vartriangle t$ is the time step, the difference coefficients are
\begin{subequations}
\begin{align}
	\alpha_{1} &= 0.0, \beta_{1} = \frac{1}{6}, \\
	\alpha_{2} &= 0.5, \beta_{2} = \frac{1}{3}, \\
	\alpha_{3} &= 0.5, \beta_{3} = \frac{1}{3}, \\
	\alpha_{4} &= 1.0, \beta_{4} = \frac{1}{6}. 
\end{align}
\end{subequations}
4 stages Runge-Kutta can be represented as
\begin{subequations}
	\begin{align}
		W^{n+1} &= \hat{L}^{BBB} \hat{L}^{FFF} \hat{L}^{BBB} \hat{L}^{FFF} W^{n}, \\
		W^{n+2} &= \hat{L}^{FFB} \hat{L}^{BBF} \hat{L}^{FFB} \hat{L}^{BBF} W^{n+1}, \\
		W^{n+3} &= \hat{L}^{BFB} \hat{L}^{FBF} \hat{L}^{BFB} \hat{L}^{FBF} W^{n+2}, \\
		W^{n+4} &= \hat{L}^{FBB} \hat{L}^{BFF} \hat{L}^{FBB} \hat{L}^{BFF} W^{n+3}, 
	\end{align}
\end{subequations}
%=============================================================
\section{Staggered-grid Finite Difference Scheme} 
%=============================================================
Staggered-grid finite-difference method is one of the most popular numerical methods to simulate elastic wave propagation. In this work, the higher-order is mainly referred to the spatial disretization.And second-order leaf-frog time marching scheme is used.

%=============================================================
\subsection{Time Discretization} 
%=============================================================
2nd-order center difference:
\newlength{\widest}
\settowidth{\widest}{$\sigma_{xy}|^{n-1/2}_{i+1/2,j+1/2,k}$}
\begin{subequations}
\begin{align}
		\makebox[\widest][l]{$\sigma_{xx}|^{n+1/2}_{i,j,k}$} &= \makebox[\widest][l]{$\sigma_{xx}|^{n-1/2}_{i,j,k}$}+\Delta t [(\lambda+2\mu)v_{x,x}+\lambda (v_{y,y}+v_{z,z})]|^n_{i,j,k} \\
		\makebox[\widest][l]{$\sigma_{yy}|^{n+1/2}_{i,j,k}$} &= \makebox[\widest][l]{$\sigma_{yy}|^{n-1/2}_{i,j,k}$} +\Delta t [(\lambda+2\mu)v_{y,y}+\lambda (v_{x,x}+v_{z,z})]|^n_{i,j,k} \\
		\makebox[\widest][l]{$\sigma_{zz}|^{n+1/2}_{i,j,k}$} &= \makebox[\widest][l]{$\sigma_{zz}|^{n-1/2}_{i,j,k}$} +\Delta t [(\lambda+2\mu)v_{z,z}+\lambda (v_{x,x}+v_{y,y})]|^n_{i,j,k} \\
		\makebox[\widest][l]{$\sigma_{xy}|^{n+1/2}_{i+1/2,j+1/2,k}$} &= \makebox[\widest][l]{$\sigma_{xy}|^{n-1/2}_{i+1/2,j+1/2,k}$}+\Delta t[\mu(v_{x,y} +v_{y,x})]|^n_{i+1/2,j+1/2,k} \\
		\makebox[\widest][l]{$\sigma_{xz}|^{n+1/2}_{i+1/2,j,k+1/2}$} &= \makebox[\widest][l]{$\sigma_{xz}|^{n-1/2}_{i+1/2,j,k+1/2}$}+\Delta t[\mu(v_{x,z} +v_{z,x})]|^n_{i+1/2,j,k+1/2} \\
		\makebox[\widest][l]{$\sigma_{yz}|^{n+1/2}_{i,j+1/2,k+1/2}$} &= \makebox[\widest][l]{$\sigma_{yz}|^{n-1/2}_{i,j+1/2,k+1/2}$}+\Delta t[\mu(v_{y,z} +v_{z,y})]|^n_{i,j+1/2,k+1/2},\\
		\makebox[\widest][l]{$v_x|^{n+1}_{i+1/2,j,k}$} &= \makebox[\widest][l]{$v_x|^{n}_{i+1/2,j,k}$}+\Delta t b_x [\sigma_{xx,x}+\sigma_{xy,y}+\sigma_{xz,z}]|^{n+1/2}_{i+1/2,j,k} \\
		\makebox[\widest][l]{$v_y|^{n+1}_{i,j+1/2,k}$} &= \makebox[\widest][l]{$v_y|^{n}_{i,j+1/2,k}$}+\Delta t b_y [\sigma_{xy,x}+\sigma_{yy,y}+\sigma_{yz,z}]|^{n+1/2}_{i,j+1/2,k} \\
		\makebox[\widest][l]{$v_z|^{n+1}_{i,j,k+1/2}$} &= \makebox[\widest][l]{$v_z|^{n}_{i,j,k+1/2}$}+\Delta t b_z [\sigma_{xz,x}+\sigma_{yz,y}+\sigma_{zz,z}]|^{n+1/2}_{i,j,k+1/2} 
\end{align}
\end{subequations}
where $b=1/\rho$.
%=============================================================
\subsection{Space Discretization} 
%=============================================================
In this work, 4th-order scheme is used(\cite{graves1996simulating}),the following is the $v_{x,x},v_{y,y},v_{z,z}$.
%\newlength{\widest}
%\settowidth{\widest}{$v_{x,y}|_{i+1/2,j+1/2,k}$}
\begin{subequations}
\begin{align}
	v_{x,x}|_{i,j,k} &= \frac{1}{\Delta h}[c_0(v_x|_{i+1/2,j,k}-v_x|_{i-1/2,j,k})-c_1(v_x|_{i+3/2,j,k}-v_x|_{i-3/2,j,k})] \\
	v_{y,y}|_{i,j,k} &= \frac{1}{\Delta h}[c_0(v_y|_{i,j+1/2,k}-v_y|_{i,j-1/2,k})-c_1(v_y|_{i,j+3/2,k}-v_y|_{i,j-3/2,k})] \\
	v_{z,z}|_{i,j,k} &= \frac{1}{\Delta h}[c_0(v_z|_{i,j,k+1/2}-v_z|_{i,j,k-1/2})-c_1(v_z|_{i,j,k+3/2}-v_z|_{i,j,k-3/2})] 
%	\makebox[\widest][l]{$v_{x,y}|_{i+1/2,j+1/2,k}$} &=\frac{1}{\Delta h}[c_0(v_x|_{i+1,j+1/2,k}-v_x|_{i,j+1/2,k})-c_1(v_x|_{i+2,j+1/2,k}-v_x|_{i-1,j+1/2,k})] \\
%	\makebox[\widest][l]{$v_{y,x}|_{i+1/2,j+1/2,k}$} &=\frac{1}{\Delta h}[c_0(v_y|_{i+1/2,j+1,k}-v_y|_{i+1/2,j,k})-c_1(v_y|_{i+1/2,j+2,k}-v_y|_{i+1/2,j-1,k})] \\
%	\makebox[\widest][l]{$v_{x,z}|_{i+1/2,j,k+1/2}$} &=\frac{1}{\Delta h}[c_0(v_x|_{i+1/2,j,k+1}-v_x|_{i+1/2,j,k})-c_1(v_x|_{i+1/2,j,k+2}-v_x|_{i+1/2,j,k-1})] \\
%	\makebox[\widest][l]{$v_{z,x}|_{i+1/2,j,k+1/2}$} &=\frac{1}{\Delta h}[c_0(v_z|_{i+1,j,k+1/2}-v_z|_{i,j,k+1/2})-c_1(v_z|_{i+2,j,k+1/2}-v_z|_{i-1,j,k+1/2})] \\
%	\makebox[\widest][l]{$v_{y,z}|_{i,j+1/2,k+1/2}$} &=\frac{1}{\Delta h}[c_0(v_y|_{i,j+1/2,k+1}-v_y|_{i,j+1/2,k})-c_1(v_y|_{i,j+1/2,k+2}-v_y|_{i,j+1/2,k-1})] \\
%	\makebox[\widest][l]{$v_{z,y}|_{i,j+1/2,k+1/2}$} &=\frac{1}{\Delta h}[c_0(v_z|_{i,j+1,k+1/2}-v_z|_{i,j,k+1/2})-c_1(v_z|_{i,j+2,k+1/2}-v_z|_{i,j-1,k+1/2})] \\
%	\makebox[\widest][l]{$\sigma_{xx,x}|_{i+1/2,j,k}$} &= \frac{1}{\Delta h}[c_0(\sigma_{xx}|_{i+1,j,k}-\sigma_{xx}|_{i,j,k})-c_1(\sigma_{xx}|_{i+2,j,k}-\sigma_{xx}|_{i-1,j,k})] \\
\end{align}
\end{subequations}
where
\begin{subequations}
\begin{align}
	c_0 &= \frac{9}{8}, \\
	c_1 &= \frac{1}{27},
\end{align}
\end{subequations}