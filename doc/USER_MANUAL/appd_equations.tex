\chapter{Appendix A}\label{appd_equations}
The solution of a physical problem is based on the solution of the governing equations. In order to adapt to the characteristics of complex shapes in discrete grids, the grids generated by different physical subregions are not the same, and the governing equations are also different in different grid systems. The governing equations in Cartesian coordinate system and curvilinear coordinate system are introduced in the following.

%=============================================================
\section{The Governing Equations in Cartesian Coordinates} \label{appd_equations}
%=============================================================
The Cartesian coordinate system is used as the background coordinate system in the multi-block grid. The high performance scalability of the numerical method under Cartesian grid is superior to the curve grid numerical method with the same precision, so the numerical method under Cartesian grid is more suitable for accurate numerical simulation.
\begin{itemize}
	\item The momentum equation:	
	\begin{align}
	\rho v_{,t} = \nabla \cdot \sigma + f ,
	\end{align}
	\item The geometric relationship:
	\begin{align}
	\varepsilon = \nabla u + (\nabla u)^T ,
	\end{align}
	\item Constitutive relation (Hooke's law):
	\begin{align}
	\sigma = C : \varepsilon .
	\end{align}
\end{itemize}
The tensor $ C $ is Lame constants, which can be reduced to $ \lambda $ and $ \mu $ in simple linear elastic media. Therefore, in the Cartesian coordinate system, the wave equation of any inhomogeneous medium can be expressed in the following different forms:
\begin{itemize}
	\item Displacement equation of second order:	
	\begin{align}
	\rho u_{i,tt} = (\lambda u_{k,k})_{,i} + (\mu u_{i,j})_{,j} + (\mu u_{j,i})_{,j} +f_i ,
	\end{align}
	\item Displacement stress system:
	\begin{align}
		\rho u_{i,tt} & = \sigma_{ij,j} + f_i \\
		\sigma_{ij} & = \lambda u_{k,k} \delta_{ij} + \mu(u_{i,j}+u_{j,i}) \\
	\end{align}
	\item First order velocity stress equation:
	\begin{align}
		\rho v_{i,t} & = \sigma_{ij,j} + f_i \\
		\sigma_{ij,t} & = \lambda v_{k,k} \delta{ij} + \mu(v_{i,j}+v_{j,i}) \\
	\end{align}
\end{itemize}
This program selects the first order velocity stress equation as the governing equation.

%=============================================================
\section{The Governing Equations in Curvilinear Coordinates} \label{appd_equations}
%=============================================================
When the study area contains undulating terrain and irregular interface, the Cartesian grid will no longer be able to discrete the medium accurately. In this case, it is necessary to introduce the curve grid and apply the fitting grid in the simulation of undulating terrain.

%=============================
\subsection{Coordinate Transformation in Curvilinear Coordinate}  
%=============================
In curvilinear coordinate system, the essence of the generation of body-fitting mesh is coordinate transformation, which maps the irregular region in the physical region to the regular region in the computational region.Suppose that the grid points in the calculation area is:$ (\xi, \eta, \zeta) $,The coordinates of the corresponding physical region are:$(x, y, z)$:
\begin{align}
	x & = x(\xi, \eta, \zeta) \\
	y & = y(\xi, \eta, \zeta) \\
	z & = z(\xi, \eta, \zeta) \\
\end{align}
According to the above mapping relations, the corresponding coordinate transformation coefficients can be solved:
\begin{align}
	x_{, \xi} & = D_\xi x , \quad x_{, \eta} = D_\eta x , \quad x_{, \zeta} = D_\zeta x , \\
	x_{, \xi} & = D_\xi y , \quad x_{, \eta} = D_\eta y , \quad x_{, \zeta} = D_\zeta y , \\
	x_{, \xi} & = D_\xi z , \quad x_{, \eta} = D_\eta z , \quad x_{, \zeta} = D_\zeta z , \\
\end{align}
By using the identity:
\begin{align}
	x_{, q}q_{,p} & = \delta_{xp} , \\
	y_{, q}q_{,p} & = \delta_{yp} , \\
	z_{, q}q_{,p} & = \delta_{zp} , \\
\end{align}
The $ q \in \xi, \eta, \zeta, p \in x,y,z$ are the Dirac delta function.The coordinate transformation coefficient can be solved:
\begin{align}
	\xi_{, x} & = (y_{,\eta}z_{,\zeta} - y_{,\zeta}z_{,\eta})J^{-1} , \\
	\xi_{, y} & = (x_{,\zeta}z_{,\eta} - x_{,\eta}z_{,\zeta})J^{-1} , \\
	\xi_{, z} & = (x_{,\eta}z_{,\zeta} - x_{,\zeta}z_{,\eta})J^{-1} , \\
	\eta_{, x} & = (y_{,\zeta}z_{,\xi} - y_{,\xi}z_{,\zeta})J^{-1} , \\
	\eta_{, y} & = (x_{,\xi}z_{,\zeta} - x_{,\zeta}z_{,\xi})J^{-1} , \\
	\eta_{, z} & = (x_{,\zeta}y_{,\xi} - x_{,\xi}z_{,\zeta})J^{-1} , \\
	\zeta_{, x} & = (y_{,\xi}z_{,\eta} - y_{,\eta}z_{,\xi})J^{-1} , \\
	\zeta_{, y} & = (x_{,\eta}z_{,\xi} - x_{,\xi}z_{,\eta})J^{-1} , \\
	\zeta_{, z} & = (x_{,\xi}z_{,\eta} - x_{,\eta}z_{,\xi})J^{-1} , \\
\end{align}
$J$ is Jacobian matrix:
\begin{center}
$J = $	
$	\begin{vmatrix}
		x_{,\xi} & x_{,\eta} & x_{,\zeta}\\
		y_{,\xi} & y_{,\eta} & y_{,\zeta}\\
		z_{,\xi} & z_{,\eta} & z_{,\zeta}
	\end{vmatrix}$
\end{center}
To ensure that mesh generation does not twist and intersect, the Jacobian determinant cannot be zero.

%=============================
\subsection{The Governing Equations in Curvilinear Coordinates}  
%=============================
In curvilinear coordinates, let $ a_1, a_2, a_3$ be the covariant basis vector for $\xi,\eta,\zeta $ in the physical region.$ a^1, a^2, a^3$ is the contravariant basis vector in the $\xi,\eta,\zeta $ direction in the physical region, and $i, j, k$ is the basis vector in the $x, y, z $ direction in the Cartesian coordinate system.So,the covariant basis vectors are:
\begin{align}
	a_1 & = x_{,\xi} \bm{i} + y_{,\xi}\bm{j} + z_{,\xi}\bm{k} , \\
	a_2 & = x_{,\eta} \bm{i} + y_{,\eta}\bm{j} + z_{,\eta}\bm{k} , \\
	a_3 & = x_{,\zeta} \bm{i} + y_{,\zeta}\bm{j} + z_{,\zeta}\bm{k} , \\
\end{align}
The contravariant basis vectors are:
\begin{align}
 	a^1 & = \xi_{,x} \bm{i} + \xi_{,y}\bm{j} + \xi_{,z}\bm{k} , \\
 	a^2 & = \eta_{,x} \bm{i} + \eta_{,y}\bm{j} + \eta_{,z}\bm{k} , \\
 	a^3 & = \zeta_{,x} \bm{i} + \zeta_{,y}\bm{j} + \zeta_{,z}\bm{k} , \\
\end{align}
According to the chain rule, the first-order speed-degree stress equation in Cartesian grid can be extended to curvilinear coordinate system. Thus the speed-stress equation in curvilinear coordinates can be formulated by the following momentum equation and Hooke's law, ignoring the source term.\\
The momentum equations are expressed as:
\begin{align}
	\rho \frac{\partial v_{x}}{\partial t} & = \xi_{, x} \sigma_{xx,\xi} + \xi_{, y} \sigma_{xy,\xi} + \xi_{, z} \sigma_{xz,\xi} \\
	& + \eta_{, x} \sigma_{xx,\eta} + \eta_{, y} \sigma_{xy,\eta} + \eta_{, z} \sigma_{xz,\eta}\\
	& + \zeta_{, x} \sigma_{xx,\zeta} + \zeta_{, y} \sigma_{xy,\zeta} + \zeta_{, z} \sigma_{xz,\zeta},\\
	\rho \frac{\partial v_{y}}{\partial t} & = \xi_{, x} \sigma_{xy,\xi} + \xi_{, y} \sigma_{yy,\xi} + \xi_{, z} \sigma_{yz,\xi} \\
	& + \eta_{, x} \sigma_{xy,\eta} + \eta_{, y} \sigma_{yy,\eta} + \eta_{, z} \sigma_{yz,\eta}\\
	& + \zeta_{, x} \sigma_{xy,\zeta} + \zeta_{, y} \sigma_{yy,\zeta} + \zeta_{, z} \sigma_{yz,\zeta},\\
	\rho \frac{\partial v_{z}}{\partial t} & = \xi_{, x} \sigma_{xz,\xi} + \xi_{, y} \sigma_{yz,\xi} + \xi_{, z} \sigma_{zz,\xi} \\
	& + \eta_{, x} \sigma_{xz,\eta} + \eta_{, y} \sigma_{yz,\eta} + \eta_{, z} \sigma_{zz,\eta}\\
	& + \zeta_{, x} \sigma_{xz,\zeta} + \zeta_{, y} \sigma_{yz,\zeta} + \zeta_{, z} \sigma_{zz,\zeta},	
\end{align}
The generalized Hooke's law is denoted as:
\begin{align}
	\frac{\partial \sigma_{xx}}{\partial t} & = (\lambda + 2\mu)\xi_{, x} v_{x,\xi} + \lambda \xi_{, y} v_{y,\xi} + \lambda \xi_{, z} v_{z,\xi} \\
	& + (\lambda + 2\mu)\eta_{, x} v_{x,\eta} + \lambda \eta_{, y} v_{y,\eta} + \lambda \eta_{, z} v_{z,\eta} \\
	& + (\lambda + 2\mu)\zeta_{, x} v_{x,\zeta} + \lambda \zeta_{, y} v_{y,\zeta} + \lambda \zeta_{, z} v_{z,\zeta},\\
	\frac{\partial \sigma_{yy}}{\partial t} & = \lambda \xi_{, x} v_{x,\xi} + (\lambda + 2\mu) \xi_{, y} v_{y,\xi} + \lambda \xi_{, z} v_{z,\xi} \\
	& + \lambda \eta_{, x} v_{x,\eta} + (\lambda + 2\mu) \eta_{, y} v_{y,\eta} + \lambda \eta_{, z} v_{z,\eta} \\
	& + \lambda \zeta_{, x} v_{x,\zeta} + (\lambda + 2\mu) \zeta_{, y} v_{y,\zeta} + \lambda \zeta_{, z} v_{z,\zeta},\\
	\frac{\partial \sigma_{zz}}{\partial t} & = \lambda \xi_{, x} v_{x,\xi} + \lambda  \xi_{, y} v_{y,\xi} + (\lambda + 2\mu) \xi_{, z} v_{z,\xi} \\
	& + \lambda \eta_{, x} v_{x,\eta} + \lambda \eta_{, y} v_{y,\eta} + (\lambda + 2\mu) \eta_{, z} v_{z,\eta} \\
	& + \lambda \zeta_{, x} v_{x,\zeta} + \lambda \zeta_{, y} v_{y,\zeta} + (\lambda + 2\mu) \zeta_{, z} v_{z,\zeta},\\
	\frac{\partial \sigma_{xy}}{\partial t} & = \mu(\xi_{,y} v_{x,\xi} + \xi_{,x}v_{y,\xi}) \\
	& + \mu(\eta_{,y} v_{x,\eta} + \eta_{,x}v_{y,\eta}) \\
	& + \mu(\zeta_{,y} v_{x,\zeta} + \zeta_{,x}v_{y,\zeta}), \\
	\frac{\partial \sigma_{xz}}{\partial t} & = \mu(\xi_{,z} v_{x,\xi} + \xi_{,x}v_{z,\xi}) \\
	& + \mu(\eta_{,z} v_{x,\eta} + \eta_{,x}v_{z,\eta}) \\
	& + \mu(\zeta_{,z} v_{x,\zeta} + \zeta_{,x}v_{z,\zeta}), \\
	\frac{\partial \sigma_{yz}}{\partial t} & = \mu(\xi_{,z} v_{y,\xi} + \xi_{,y}v_{z,\xi}) \\
	& + \mu(\eta_{,z} v_{y,\eta} + \eta_{,y}v_{z,\eta}) \\
	& + \mu(\zeta_{,z} v_{y,\zeta} + \zeta_{,y}v_{z,\zeta}).
\end{align}